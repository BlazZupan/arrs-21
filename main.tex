\documentclass[11pt,a4paper]{article}
\usepackage[top=2cm,left=2cm,right=2cm,footskip=0.75in]{geometry}
\usepackage{graphicx}
\usepackage{amsmath,amssymb}
\usepackage{url}
\usepackage[utf8x]{inputenc}
\usepackage[document]{ragged2e}
\setcounter{section}{26}  % this is a section 27 of the proposal form

% \setlength{\oddsidemargin}{0.25in}
% \setlength{\textwidth}{6.5in}
% \setlength{\topmargin}{0in}
% \setlength{\textheight}{8.5in}

\renewcommand{\bold}{\textbf}

\begin{document}

\title{Computational Toolbox for Discovery of Prognostic Markers in Survival Analysis from Gene Expression Data}
\author{}
\date{\today}
\maketitle

\section{Scientific background, problem identification and objectives of the proposed research}
\subsection{Scientific Background}
\subsection{Problem Description}
\subsection{Project Aims}
\subsection{Anticipated Results}
The expected principal results of this project are:
\begin{enumerate}
	\item one
	\item two
	\item three
\end{enumerate}

\section{State-of-the-art in the proposed field of research and survey of the relevant literature}

Intro paragraph.

\subsection{Computation Methods for Gene Markers Identification in Survival Analysis}
\subsection{Visual Data Analysis}
\subsection{Toolboxes}

\section{Detailed Description of the Work Programme}

\subsection{Research and Application Challenges, Objectives and Tasks}

\subsubsection{Challenges}

\subsubsection{Objectives}
The principal research objective of this project is to:
\begin{enumerate}
	\item develop a computational pipeline
	\item design visual interfaces for explorative data analysis
	\item prove that a combination of computational methods and the implemented human-computer interfaces can significantly advace state of the art of the biomarker discovery in survival analysis
\end{enumerate}

\subsubsection{Project Tasks}

The project will be organized around a following set of tasks:
\begin{description}
	\item[T0] \bold{Setting-up of the collaborative environment.} Wiki pages enhanced with several tools for collaborative work will be used as a platform to organize project document, organize and store working notes of the meetings, and post messages and announcements.
	\item[T1] \bold{Data acquisition and organization.} The project will use a number of different data sets coming from our own experiments or from other studies already published and publicly available, which we will, to allow interoperability, organize under the same software platform and provide means for their simple access.
	\item[T2] \bold{Development of data mining and bioinformatics for survival biomarker discovery.} In particular,
	\begin{description}
		\item[T2.1] Techniques for ...
		\item[T2.2] Techniques for ...
		\item[T2.3] Groth-curve ...
	\end{description}
	\item[T3] \bold{Design of visual interfaces for explorative analysis of survival data and biomarker discovery.}
	\item[T4] \bold{Implementation and Integration.} Computational techniques we will develop in the project will be implemented within an open-source data mining environment Orange\footnote{\url{http://orangedatamining.com})}.
	\item[T5] \bold{Experimental validation.} Experiments on synthetic and real data sets, comparison with results from the literature.
	\item[T6] \bold{Dissemination of results.} including publishing of the implementation of the developed methods under General Public License (GPL), writing and web-publishing of relevant documentation with working examples, and dissemination in terms of presentation at relevant conferences and journal publications.
\end{description}

\subsection{Research Design and Methods}
\subsubsection{Overview}
\subsubsection{Material and Data Sets}
\subsubsection{Computational Approaches, Data Mining and Bioinformatics}
\subsubsection{Software Implementation}
\subsubsection{Experimental Validation}

\section{Available research equipment over 5.000 €}

\section{Project management}
% Detailed implementation plan and timetable
% gantt
% risks and remedies

\section{Introduction}

\end{document}